\section{Mistake Bound Model of Learning}\label{sec:q3}

In both the questions below, we will consider functions defined over
$n$ Boolean features. That is, each example in our learning problem
is a $n$-dimensional vector from $\{0, 1\}^n$. We will use the
symbol $\bx$ to denote an example and $\bx_i$ denotes its $i^{th}$
element.  (We will assume that there is no noise involved.)

For all questions below, it is not enough to just state the
answer. You need to justify your answer with a short proof.


\begin{enumerate}
\item Consider the concept class $\mathcal{C}_1$ defined as follows:
  Each element of $\mathcal{C}_1$ is defined using a fixed instance
  $\bz \in \{0, 1\}^n$ as follows:
  \begin{equation*}
    f_\bz(\bx) = \begin{cases}
      1 & \bx = \bz \\
      0 & \bx \ne \bz.
    \end{cases}
  \end{equation*}
  That is, the function $f_\bz$ predicts $1$ if, and only if, the
  input to the function is $\bz$.

  Our goal is to come up with a mistake bound algorithm that will
  learn any function $f\in\mathcal{C}_1$.
  
  \begin{enumerate}
  \item~[5 points] Determine $\vert\mathcal{C}_1\vert$, the size of
    concept class.
    
  \item~[15 points] Write a mistake bound learning algorithm for
    this concept class that makes no more than {\em one} mistake on
    any sequence of examples presented to it. Please write the
    algorithm concisely in the form of pseudocode.

    Prove the mistake bound for this algorithm.

  \end{enumerate}

\item Suppose we have a concept class $\mathcal{C}_2$ that consists
  of exactly $n$ functions $\{f_1, f_2, \cdots, f_n\}$, where each
  function $f_i$ is defined as follows:
  % 
  \begin{equation*}
    f_i(\bx) = \bx_i.
  \end{equation*}
  That is, the function $f_i$ returns the value of the $i^{th}$
  feature.
  
  \begin{enumerate}
  \item~[5 points] How many mistakes will the algorithm
    \textbf{CON} from class make on any function from this concept
    class?
  \item~[5 points] How many mistakes will the Halving algorithm make
    on any function from this concept class?
  \end{enumerate}
  
\end{enumerate}



%%% Local Variables:
%%% mode: latex
%%% TeX-master: "hw2"
%%% End:
